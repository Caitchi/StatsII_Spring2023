\documentclass[12pt,letterpaper]{article}
\usepackage{graphicx,textcomp}
\usepackage{natbib}
\usepackage{setspace}
\usepackage{fullpage}
\usepackage{color}
\usepackage[reqno]{amsmath}
\usepackage{amsthm}
\usepackage{fancyvrb}
\usepackage{amssymb,enumerate}
\usepackage[all]{xy}
\usepackage{endnotes}
\usepackage{lscape}
\newtheorem{com}{Comment}
\usepackage{float}
\usepackage{hyperref}
\newtheorem{lem} {Lemma}
\newtheorem{prop}{Proposition}
\newtheorem{thm}{Theorem}
\newtheorem{defn}{Definition}
\newtheorem{cor}{Corollary}
\newtheorem{obs}{Observation}
\usepackage[compact]{titlesec}
\usepackage{dcolumn}
\usepackage{tikz}
\usetikzlibrary{arrows}
\usepackage{multirow}
\usepackage{xcolor}
\newcolumntype{.}{D{.}{.}{-1}}
\newcolumntype{d}[1]{D{.}{.}{#1}}
\definecolor{light-gray}{gray}{0.65}
\usepackage{url}
\usepackage{listings}
\usepackage{color}

\definecolor{codegreen}{rgb}{0,0.6,0}
\definecolor{codegray}{rgb}{0.5,0.5,0.5}
\definecolor{codepurple}{rgb}{0.58,0,0.82}
\definecolor{backcolour}{rgb}{0.95,0.95,0.92}

\lstdefinestyle{mystyle}{
	backgroundcolor=\color{backcolour},   
	commentstyle=\color{codegreen},
	keywordstyle=\color{magenta},
	numberstyle=\tiny\color{codegray},
	stringstyle=\color{codepurple},
	basicstyle=\footnotesize,
	breakatwhitespace=false,         
	breaklines=true,                 
	captionpos=b,                    
	keepspaces=true,                 
	numbers=left,                    
	numbersep=5pt,                  
	showspaces=false,                
	showstringspaces=false,
	showtabs=false,                  
	tabsize=2
}
\lstset{style=mystyle}
\newcommand{\Sref}[1]{Section~\ref{#1}}
\newtheorem{hyp}{Hypothesis}

\title{Problem Set 1: Applied Stats II}
\date{Due: February 19, 2023}
\author{Caitlín Cooney}


\begin{document}
	\maketitle
	\vspace{.25cm}
\section*{Question 1} 
\vspace{.25cm}


\noindent Using \texttt{R} generate 1,000 Cauchy random variables (\texttt{rcauchy(1000, location = 0, scale = 1)}) and perform the test (remember, use the same seed, something like \texttt{set.seed(123)}, whenever you're generating your own data).\\

\begin{lstlisting}[language=R]
	set.seed(123)data <- (rcauchy(1000, location = 0, scale = 1)) \end{lstlisting}
	
\noindent  Write an \texttt{R} function that implements the Kolmogorov-Smirnov test where the reference distribution is normal. \\
	

\begin{lstlisting}[language=R]
 	kolsmir.test <- function(data) { 
		
		# Create empirical distribution of observed data  
		ECDF <- ecdf(data)  
		empiricalCDF <- ECDF(data)    
		
		# Generate test statistic  
		D <- max(abs(empiricalCDF - pnorm(data)))    
		
		# Calculate p-value  
		p.value <- 1 - pnorm(sqrt(length(data)) * D)    
		
		return(list(D = D, p.value = p.value))
	}\end{lstlisting}


\noindent State your null and alternate hypotheses: \\

\indent H0: The data comes from the specified distribution.\\

\indent H1: At least one value does not match the specified distribution.\\


\begin{lstlisting}[language=R]
	# Run the function
	kolsmir.test(data) \end{lstlisting}

\newpage
\noindent The results are: \begin{verbatim}
	$D
	[1] 0.1347281
	$p.value
	[1] 1.019963e-0}\end{verbatim}

\noindent  As D is not greater than the critical value, we fail to reject the null hypothesis that  the data comes from the specified distribution (or in other words, that P = P0).\\

\vspace{3in}

\newpage

\section*{Question 2}
\noindent Estimate an OLS regression in \texttt{R} that uses the Newton-Raphson algorithm (specifically \texttt{BFGS}, which is a quasi-Newton method).  
\vspace{.5cm}


\begin{lstlisting}[language=R]
set.seed (123)
data <- data.frame(x = runif(200, 1, 10))
data$y <- 0 + 2.75*data$x + rnorm(200, 0, 1.5) \end{lstlisting}

\begin{lstlisting}[language=R]
	linear.lik <- function(theta, y, X) {  
		n <- nrow(X)  
		k <- ncol(X)  
		beta <- theta [1:k]  
		sigma2<- theta [k+1] ^2  
		e <- y-X%*%beta  
		logl<- -.5*n*log( 2*pi )-.5*n*log(sigma2)-((t(e)%*%e)/(2*sigma2))
		return(-logl)
	}
	 \end{lstlisting}
		
\noindent Find parameters that specify the point. \\


\begin{lstlisting}[language=R]
	linear.MLE <- optim(fn = linear.lik, par = c(1, 1, 1), 
									hessian = TRUE, y = data$y , X = cbind(1, data$x),  
									method = "BFGS")
	
	linear.MLE$par\end{lstlisting}


\noindent The results show: \\

\begin{verbatim}
[1]  0.1398324  2.7265559 -1.4390716 \end{verbatim}


\noindent Show that you get the equivalent results to using \texttt{lm}. \\

\begin{lstlisting}[language=R]
	summary(lm(y~x, data)) \end{lstlisting}

\noindent The results show: \\

 \begin{verbatim} 
 	Coefficients:            
 	Estimate Std. Error t value Pr(>|t|)    
 	(Intercept)  	0.13919    0.25276   0.551    0.582   
 	x            			2.72670    0.04159  65.564   <2e-16 ***\end{verbatim}
 
 \noindent As expected, the estimates for \texttt{lm} are the same as the parameters found using the Newton-Raphson algorithm, because ordinary least squares is equivalent to maximum likelihood for a linear model, so it makes sense that \texttt{lm} would give us the same answers. \\
 

\end{document}
